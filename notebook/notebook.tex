
% Default to the notebook output style

    


% Inherit from the specified cell style.




    
\documentclass[11pt]{article}

    
    
    \usepackage[T1]{fontenc}
    % Nicer default font (+ math font) than Computer Modern for most use cases
    \usepackage{mathpazo}

    % Basic figure setup, for now with no caption control since it's done
    % automatically by Pandoc (which extracts ![](path) syntax from Markdown).
    \usepackage{graphicx}
    % We will generate all images so they have a width \maxwidth. This means
    % that they will get their normal width if they fit onto the page, but
    % are scaled down if they would overflow the margins.
    \makeatletter
    \def\maxwidth{\ifdim\Gin@nat@width>\linewidth\linewidth
    \else\Gin@nat@width\fi}
    \makeatother
    \let\Oldincludegraphics\includegraphics
    % Set max figure width to be 80% of text width, for now hardcoded.
    \renewcommand{\includegraphics}[1]{\Oldincludegraphics[width=.8\maxwidth]{#1}}
    % Ensure that by default, figures have no caption (until we provide a
    % proper Figure object with a Caption API and a way to capture that
    % in the conversion process - todo).
    \usepackage{caption}
    \DeclareCaptionLabelFormat{nolabel}{}
    \captionsetup{labelformat=nolabel}

    \usepackage{adjustbox} % Used to constrain images to a maximum size 
    \usepackage{xcolor} % Allow colors to be defined
    \usepackage{enumerate} % Needed for markdown enumerations to work
    \usepackage{geometry} % Used to adjust the document margins
    \usepackage{amsmath} % Equations
    \usepackage{amssymb} % Equations
    \usepackage{textcomp} % defines textquotesingle
    % Hack from http://tex.stackexchange.com/a/47451/13684:
    \AtBeginDocument{%
        \def\PYZsq{\textquotesingle}% Upright quotes in Pygmentized code
    }
    \usepackage{upquote} % Upright quotes for verbatim code
    \usepackage{eurosym} % defines \euro
    \usepackage[mathletters]{ucs} % Extended unicode (utf-8) support
    \usepackage[utf8x]{inputenc} % Allow utf-8 characters in the tex document
    \usepackage{fancyvrb} % verbatim replacement that allows latex
    \usepackage{grffile} % extends the file name processing of package graphics 
                         % to support a larger range 
    % The hyperref package gives us a pdf with properly built
    % internal navigation ('pdf bookmarks' for the table of contents,
    % internal cross-reference links, web links for URLs, etc.)
    \usepackage{hyperref}
    \usepackage{longtable} % longtable support required by pandoc >1.10
    \usepackage{booktabs}  % table support for pandoc > 1.12.2
    \usepackage[inline]{enumitem} % IRkernel/repr support (it uses the enumerate* environment)
    \usepackage[normalem]{ulem} % ulem is needed to support strikethroughs (\sout)
                                % normalem makes italics be italics, not underlines
    

    
    
    % Colors for the hyperref package
    \definecolor{urlcolor}{rgb}{0,.145,.698}
    \definecolor{linkcolor}{rgb}{.71,0.21,0.01}
    \definecolor{citecolor}{rgb}{.12,.54,.11}

    % ANSI colors
    \definecolor{ansi-black}{HTML}{3E424D}
    \definecolor{ansi-black-intense}{HTML}{282C36}
    \definecolor{ansi-red}{HTML}{E75C58}
    \definecolor{ansi-red-intense}{HTML}{B22B31}
    \definecolor{ansi-green}{HTML}{00A250}
    \definecolor{ansi-green-intense}{HTML}{007427}
    \definecolor{ansi-yellow}{HTML}{DDB62B}
    \definecolor{ansi-yellow-intense}{HTML}{B27D12}
    \definecolor{ansi-blue}{HTML}{208FFB}
    \definecolor{ansi-blue-intense}{HTML}{0065CA}
    \definecolor{ansi-magenta}{HTML}{D160C4}
    \definecolor{ansi-magenta-intense}{HTML}{A03196}
    \definecolor{ansi-cyan}{HTML}{60C6C8}
    \definecolor{ansi-cyan-intense}{HTML}{258F8F}
    \definecolor{ansi-white}{HTML}{C5C1B4}
    \definecolor{ansi-white-intense}{HTML}{A1A6B2}

    % commands and environments needed by pandoc snippets
    % extracted from the output of `pandoc -s`
    \providecommand{\tightlist}{%
      \setlength{\itemsep}{0pt}\setlength{\parskip}{0pt}}
    \DefineVerbatimEnvironment{Highlighting}{Verbatim}{commandchars=\\\{\}}
    % Add ',fontsize=\small' for more characters per line
    \newenvironment{Shaded}{}{}
    \newcommand{\KeywordTok}[1]{\textcolor[rgb]{0.00,0.44,0.13}{\textbf{{#1}}}}
    \newcommand{\DataTypeTok}[1]{\textcolor[rgb]{0.56,0.13,0.00}{{#1}}}
    \newcommand{\DecValTok}[1]{\textcolor[rgb]{0.25,0.63,0.44}{{#1}}}
    \newcommand{\BaseNTok}[1]{\textcolor[rgb]{0.25,0.63,0.44}{{#1}}}
    \newcommand{\FloatTok}[1]{\textcolor[rgb]{0.25,0.63,0.44}{{#1}}}
    \newcommand{\CharTok}[1]{\textcolor[rgb]{0.25,0.44,0.63}{{#1}}}
    \newcommand{\StringTok}[1]{\textcolor[rgb]{0.25,0.44,0.63}{{#1}}}
    \newcommand{\CommentTok}[1]{\textcolor[rgb]{0.38,0.63,0.69}{\textit{{#1}}}}
    \newcommand{\OtherTok}[1]{\textcolor[rgb]{0.00,0.44,0.13}{{#1}}}
    \newcommand{\AlertTok}[1]{\textcolor[rgb]{1.00,0.00,0.00}{\textbf{{#1}}}}
    \newcommand{\FunctionTok}[1]{\textcolor[rgb]{0.02,0.16,0.49}{{#1}}}
    \newcommand{\RegionMarkerTok}[1]{{#1}}
    \newcommand{\ErrorTok}[1]{\textcolor[rgb]{1.00,0.00,0.00}{\textbf{{#1}}}}
    \newcommand{\NormalTok}[1]{{#1}}
    
    % Additional commands for more recent versions of Pandoc
    \newcommand{\ConstantTok}[1]{\textcolor[rgb]{0.53,0.00,0.00}{{#1}}}
    \newcommand{\SpecialCharTok}[1]{\textcolor[rgb]{0.25,0.44,0.63}{{#1}}}
    \newcommand{\VerbatimStringTok}[1]{\textcolor[rgb]{0.25,0.44,0.63}{{#1}}}
    \newcommand{\SpecialStringTok}[1]{\textcolor[rgb]{0.73,0.40,0.53}{{#1}}}
    \newcommand{\ImportTok}[1]{{#1}}
    \newcommand{\DocumentationTok}[1]{\textcolor[rgb]{0.73,0.13,0.13}{\textit{{#1}}}}
    \newcommand{\AnnotationTok}[1]{\textcolor[rgb]{0.38,0.63,0.69}{\textbf{\textit{{#1}}}}}
    \newcommand{\CommentVarTok}[1]{\textcolor[rgb]{0.38,0.63,0.69}{\textbf{\textit{{#1}}}}}
    \newcommand{\VariableTok}[1]{\textcolor[rgb]{0.10,0.09,0.49}{{#1}}}
    \newcommand{\ControlFlowTok}[1]{\textcolor[rgb]{0.00,0.44,0.13}{\textbf{{#1}}}}
    \newcommand{\OperatorTok}[1]{\textcolor[rgb]{0.40,0.40,0.40}{{#1}}}
    \newcommand{\BuiltInTok}[1]{{#1}}
    \newcommand{\ExtensionTok}[1]{{#1}}
    \newcommand{\PreprocessorTok}[1]{\textcolor[rgb]{0.74,0.48,0.00}{{#1}}}
    \newcommand{\AttributeTok}[1]{\textcolor[rgb]{0.49,0.56,0.16}{{#1}}}
    \newcommand{\InformationTok}[1]{\textcolor[rgb]{0.38,0.63,0.69}{\textbf{\textit{{#1}}}}}
    \newcommand{\WarningTok}[1]{\textcolor[rgb]{0.38,0.63,0.69}{\textbf{\textit{{#1}}}}}
    
    
    % Define a nice break command that doesn't care if a line doesn't already
    % exist.
    \def\br{\hspace*{\fill} \\* }
    % Math Jax compatability definitions
    \def\gt{>}
    \def\lt{<}
    % Document parameters
    \title{Austin\_Zadoks\_Project\_Colab}
    
    
    

    % Pygments definitions
    
\makeatletter
\def\PY@reset{\let\PY@it=\relax \let\PY@bf=\relax%
    \let\PY@ul=\relax \let\PY@tc=\relax%
    \let\PY@bc=\relax \let\PY@ff=\relax}
\def\PY@tok#1{\csname PY@tok@#1\endcsname}
\def\PY@toks#1+{\ifx\relax#1\empty\else%
    \PY@tok{#1}\expandafter\PY@toks\fi}
\def\PY@do#1{\PY@bc{\PY@tc{\PY@ul{%
    \PY@it{\PY@bf{\PY@ff{#1}}}}}}}
\def\PY#1#2{\PY@reset\PY@toks#1+\relax+\PY@do{#2}}

\expandafter\def\csname PY@tok@gd\endcsname{\def\PY@tc##1{\textcolor[rgb]{0.63,0.00,0.00}{##1}}}
\expandafter\def\csname PY@tok@gu\endcsname{\let\PY@bf=\textbf\def\PY@tc##1{\textcolor[rgb]{0.50,0.00,0.50}{##1}}}
\expandafter\def\csname PY@tok@gt\endcsname{\def\PY@tc##1{\textcolor[rgb]{0.00,0.27,0.87}{##1}}}
\expandafter\def\csname PY@tok@gs\endcsname{\let\PY@bf=\textbf}
\expandafter\def\csname PY@tok@gr\endcsname{\def\PY@tc##1{\textcolor[rgb]{1.00,0.00,0.00}{##1}}}
\expandafter\def\csname PY@tok@cm\endcsname{\let\PY@it=\textit\def\PY@tc##1{\textcolor[rgb]{0.25,0.50,0.50}{##1}}}
\expandafter\def\csname PY@tok@vg\endcsname{\def\PY@tc##1{\textcolor[rgb]{0.10,0.09,0.49}{##1}}}
\expandafter\def\csname PY@tok@vi\endcsname{\def\PY@tc##1{\textcolor[rgb]{0.10,0.09,0.49}{##1}}}
\expandafter\def\csname PY@tok@vm\endcsname{\def\PY@tc##1{\textcolor[rgb]{0.10,0.09,0.49}{##1}}}
\expandafter\def\csname PY@tok@mh\endcsname{\def\PY@tc##1{\textcolor[rgb]{0.40,0.40,0.40}{##1}}}
\expandafter\def\csname PY@tok@cs\endcsname{\let\PY@it=\textit\def\PY@tc##1{\textcolor[rgb]{0.25,0.50,0.50}{##1}}}
\expandafter\def\csname PY@tok@ge\endcsname{\let\PY@it=\textit}
\expandafter\def\csname PY@tok@vc\endcsname{\def\PY@tc##1{\textcolor[rgb]{0.10,0.09,0.49}{##1}}}
\expandafter\def\csname PY@tok@il\endcsname{\def\PY@tc##1{\textcolor[rgb]{0.40,0.40,0.40}{##1}}}
\expandafter\def\csname PY@tok@go\endcsname{\def\PY@tc##1{\textcolor[rgb]{0.53,0.53,0.53}{##1}}}
\expandafter\def\csname PY@tok@cp\endcsname{\def\PY@tc##1{\textcolor[rgb]{0.74,0.48,0.00}{##1}}}
\expandafter\def\csname PY@tok@gi\endcsname{\def\PY@tc##1{\textcolor[rgb]{0.00,0.63,0.00}{##1}}}
\expandafter\def\csname PY@tok@gh\endcsname{\let\PY@bf=\textbf\def\PY@tc##1{\textcolor[rgb]{0.00,0.00,0.50}{##1}}}
\expandafter\def\csname PY@tok@ni\endcsname{\let\PY@bf=\textbf\def\PY@tc##1{\textcolor[rgb]{0.60,0.60,0.60}{##1}}}
\expandafter\def\csname PY@tok@nl\endcsname{\def\PY@tc##1{\textcolor[rgb]{0.63,0.63,0.00}{##1}}}
\expandafter\def\csname PY@tok@nn\endcsname{\let\PY@bf=\textbf\def\PY@tc##1{\textcolor[rgb]{0.00,0.00,1.00}{##1}}}
\expandafter\def\csname PY@tok@no\endcsname{\def\PY@tc##1{\textcolor[rgb]{0.53,0.00,0.00}{##1}}}
\expandafter\def\csname PY@tok@na\endcsname{\def\PY@tc##1{\textcolor[rgb]{0.49,0.56,0.16}{##1}}}
\expandafter\def\csname PY@tok@nb\endcsname{\def\PY@tc##1{\textcolor[rgb]{0.00,0.50,0.00}{##1}}}
\expandafter\def\csname PY@tok@nc\endcsname{\let\PY@bf=\textbf\def\PY@tc##1{\textcolor[rgb]{0.00,0.00,1.00}{##1}}}
\expandafter\def\csname PY@tok@nd\endcsname{\def\PY@tc##1{\textcolor[rgb]{0.67,0.13,1.00}{##1}}}
\expandafter\def\csname PY@tok@ne\endcsname{\let\PY@bf=\textbf\def\PY@tc##1{\textcolor[rgb]{0.82,0.25,0.23}{##1}}}
\expandafter\def\csname PY@tok@nf\endcsname{\def\PY@tc##1{\textcolor[rgb]{0.00,0.00,1.00}{##1}}}
\expandafter\def\csname PY@tok@si\endcsname{\let\PY@bf=\textbf\def\PY@tc##1{\textcolor[rgb]{0.73,0.40,0.53}{##1}}}
\expandafter\def\csname PY@tok@s2\endcsname{\def\PY@tc##1{\textcolor[rgb]{0.73,0.13,0.13}{##1}}}
\expandafter\def\csname PY@tok@nt\endcsname{\let\PY@bf=\textbf\def\PY@tc##1{\textcolor[rgb]{0.00,0.50,0.00}{##1}}}
\expandafter\def\csname PY@tok@nv\endcsname{\def\PY@tc##1{\textcolor[rgb]{0.10,0.09,0.49}{##1}}}
\expandafter\def\csname PY@tok@s1\endcsname{\def\PY@tc##1{\textcolor[rgb]{0.73,0.13,0.13}{##1}}}
\expandafter\def\csname PY@tok@dl\endcsname{\def\PY@tc##1{\textcolor[rgb]{0.73,0.13,0.13}{##1}}}
\expandafter\def\csname PY@tok@ch\endcsname{\let\PY@it=\textit\def\PY@tc##1{\textcolor[rgb]{0.25,0.50,0.50}{##1}}}
\expandafter\def\csname PY@tok@m\endcsname{\def\PY@tc##1{\textcolor[rgb]{0.40,0.40,0.40}{##1}}}
\expandafter\def\csname PY@tok@gp\endcsname{\let\PY@bf=\textbf\def\PY@tc##1{\textcolor[rgb]{0.00,0.00,0.50}{##1}}}
\expandafter\def\csname PY@tok@sh\endcsname{\def\PY@tc##1{\textcolor[rgb]{0.73,0.13,0.13}{##1}}}
\expandafter\def\csname PY@tok@ow\endcsname{\let\PY@bf=\textbf\def\PY@tc##1{\textcolor[rgb]{0.67,0.13,1.00}{##1}}}
\expandafter\def\csname PY@tok@sx\endcsname{\def\PY@tc##1{\textcolor[rgb]{0.00,0.50,0.00}{##1}}}
\expandafter\def\csname PY@tok@bp\endcsname{\def\PY@tc##1{\textcolor[rgb]{0.00,0.50,0.00}{##1}}}
\expandafter\def\csname PY@tok@c1\endcsname{\let\PY@it=\textit\def\PY@tc##1{\textcolor[rgb]{0.25,0.50,0.50}{##1}}}
\expandafter\def\csname PY@tok@fm\endcsname{\def\PY@tc##1{\textcolor[rgb]{0.00,0.00,1.00}{##1}}}
\expandafter\def\csname PY@tok@o\endcsname{\def\PY@tc##1{\textcolor[rgb]{0.40,0.40,0.40}{##1}}}
\expandafter\def\csname PY@tok@kc\endcsname{\let\PY@bf=\textbf\def\PY@tc##1{\textcolor[rgb]{0.00,0.50,0.00}{##1}}}
\expandafter\def\csname PY@tok@c\endcsname{\let\PY@it=\textit\def\PY@tc##1{\textcolor[rgb]{0.25,0.50,0.50}{##1}}}
\expandafter\def\csname PY@tok@mf\endcsname{\def\PY@tc##1{\textcolor[rgb]{0.40,0.40,0.40}{##1}}}
\expandafter\def\csname PY@tok@err\endcsname{\def\PY@bc##1{\setlength{\fboxsep}{0pt}\fcolorbox[rgb]{1.00,0.00,0.00}{1,1,1}{\strut ##1}}}
\expandafter\def\csname PY@tok@mb\endcsname{\def\PY@tc##1{\textcolor[rgb]{0.40,0.40,0.40}{##1}}}
\expandafter\def\csname PY@tok@ss\endcsname{\def\PY@tc##1{\textcolor[rgb]{0.10,0.09,0.49}{##1}}}
\expandafter\def\csname PY@tok@sr\endcsname{\def\PY@tc##1{\textcolor[rgb]{0.73,0.40,0.53}{##1}}}
\expandafter\def\csname PY@tok@mo\endcsname{\def\PY@tc##1{\textcolor[rgb]{0.40,0.40,0.40}{##1}}}
\expandafter\def\csname PY@tok@kd\endcsname{\let\PY@bf=\textbf\def\PY@tc##1{\textcolor[rgb]{0.00,0.50,0.00}{##1}}}
\expandafter\def\csname PY@tok@mi\endcsname{\def\PY@tc##1{\textcolor[rgb]{0.40,0.40,0.40}{##1}}}
\expandafter\def\csname PY@tok@kn\endcsname{\let\PY@bf=\textbf\def\PY@tc##1{\textcolor[rgb]{0.00,0.50,0.00}{##1}}}
\expandafter\def\csname PY@tok@cpf\endcsname{\let\PY@it=\textit\def\PY@tc##1{\textcolor[rgb]{0.25,0.50,0.50}{##1}}}
\expandafter\def\csname PY@tok@kr\endcsname{\let\PY@bf=\textbf\def\PY@tc##1{\textcolor[rgb]{0.00,0.50,0.00}{##1}}}
\expandafter\def\csname PY@tok@s\endcsname{\def\PY@tc##1{\textcolor[rgb]{0.73,0.13,0.13}{##1}}}
\expandafter\def\csname PY@tok@kp\endcsname{\def\PY@tc##1{\textcolor[rgb]{0.00,0.50,0.00}{##1}}}
\expandafter\def\csname PY@tok@w\endcsname{\def\PY@tc##1{\textcolor[rgb]{0.73,0.73,0.73}{##1}}}
\expandafter\def\csname PY@tok@kt\endcsname{\def\PY@tc##1{\textcolor[rgb]{0.69,0.00,0.25}{##1}}}
\expandafter\def\csname PY@tok@sc\endcsname{\def\PY@tc##1{\textcolor[rgb]{0.73,0.13,0.13}{##1}}}
\expandafter\def\csname PY@tok@sb\endcsname{\def\PY@tc##1{\textcolor[rgb]{0.73,0.13,0.13}{##1}}}
\expandafter\def\csname PY@tok@sa\endcsname{\def\PY@tc##1{\textcolor[rgb]{0.73,0.13,0.13}{##1}}}
\expandafter\def\csname PY@tok@k\endcsname{\let\PY@bf=\textbf\def\PY@tc##1{\textcolor[rgb]{0.00,0.50,0.00}{##1}}}
\expandafter\def\csname PY@tok@se\endcsname{\let\PY@bf=\textbf\def\PY@tc##1{\textcolor[rgb]{0.73,0.40,0.13}{##1}}}
\expandafter\def\csname PY@tok@sd\endcsname{\let\PY@it=\textit\def\PY@tc##1{\textcolor[rgb]{0.73,0.13,0.13}{##1}}}

\def\PYZbs{\char`\\}
\def\PYZus{\char`\_}
\def\PYZob{\char`\{}
\def\PYZcb{\char`\}}
\def\PYZca{\char`\^}
\def\PYZam{\char`\&}
\def\PYZlt{\char`\<}
\def\PYZgt{\char`\>}
\def\PYZsh{\char`\#}
\def\PYZpc{\char`\%}
\def\PYZdl{\char`\$}
\def\PYZhy{\char`\-}
\def\PYZsq{\char`\'}
\def\PYZdq{\char`\"}
\def\PYZti{\char`\~}
% for compatibility with earlier versions
\def\PYZat{@}
\def\PYZlb{[}
\def\PYZrb{]}
\makeatother


    % Exact colors from NB
    \definecolor{incolor}{rgb}{0.0, 0.0, 0.5}
    \definecolor{outcolor}{rgb}{0.545, 0.0, 0.0}



    
    % Prevent overflowing lines due to hard-to-break entities
    \sloppy 
    % Setup hyperref package
    \hypersetup{
      breaklinks=true,  % so long urls are correctly broken across lines
      colorlinks=true,
      urlcolor=urlcolor,
      linkcolor=linkcolor,
      citecolor=citecolor,
      }
    % Slightly bigger margins than the latex defaults
    
    \geometry{verbose,tmargin=1in,bmargin=1in,lmargin=1in,rmargin=1in}
    
    

    \begin{document}
    
    
    \maketitle
    
    

    
    \subsection{TODO}\label{todo}

\begin{itemize}
\tightlist
\item
  REST + Flask
\item
  Pymatgen + Mongo
\item
  Flask + Mongo
\item
  List of descriptors
\item
  List of comparitors
\item
  Shape up the lit. review
\end{itemize}

    \subsection{Structure prototyping / description
methods}\label{structure-prototyping-description-methods}

    \subsubsection{Strukturbericht 1931}\label{strukturbericht-1931}

    \subsubsection{Herman-Mauguin (ICT) Space Group 1928, 1931,
1935}\label{herman-mauguin-ict-space-group-1928-1931-1935}

    \subsubsection{Pearson Symbol 1958}\label{pearson-symbol-1958}

    \subsubsection{Steinhardt Bond Orientational Order Parameter
1983}\label{steinhardt-bond-orientational-order-parameter-1983}

    \subsubsection{Parthé Standardization
1984}\label{parthuxe9-standardization-1984}

    \subsubsection{Lima De Faria Definitions of Structural Similarity
1990}\label{lima-de-faria-definitions-of-structural-similarity-1990}

    \subsubsection{Bartók SOAP 2013}\label{bartuxf3k-soap-2013}

    \subsubsection{Zimmermann 2017}\label{zimmermann-2017}

    \subsection{Databases}\label{databases}

    \subsubsection{Pauling}\label{pauling}

    \subsubsection{ICSD}\label{icsd}

    \subsubsection{AFLOW Mehl 2017}\label{aflow-mehl-2017}

\begin{itemize}
\tightlist
\item
  Space Group
\item
  Pearson Symbol
\item
  Strukturbericht
\item
  Chemical Symbol
\item
  AFLOW Prototype Index
\end{itemize}

    \subsubsection{Materials Project}\label{materials-project}

\begin{itemize}
\tightlist
\item
  Hermann Mauguin Space Group
\item
  Similarity with up to 5 other materials
\item
  48x1 vector fingerprint \(\mathbf{v^{site}}\) calculated for each site
  using matminer (python)

  \begin{itemize}
  \tightlist
  \item
    \(q_n\) is the Steinhardt bond orientational order parameter of
    order \(n\)
  \end{itemize}
\item
  structure fingerprint is a vector of the min, max, mean , std of each
  entry of the sites
\item
  structure is\\
\item
  N. E. R. Zimmermann, M. K. Horton, A. Jain, M. Haranczyk, Front.
  Mater., 4, 34, (2017)
\item
  N. E. R. Zimmermann, A. Jain, in preparation (2018)
\end{itemize}

    \section{Literature}\label{literature}

    \subsubsection{Niggli 1928}\label{niggli-1928}

    \subsubsection{\texorpdfstring{\emph{Strukturbericht}
1931}{Strukturbericht 1931}}\label{strukturbericht-1931}

    \subsubsection{Pearson 1958}\label{pearson-1958}

    \subsubsection{Buerger 1960}\label{buerger-1960}

    \subsubsection{Schwarzenbach 1963}\label{schwarzenbach-1963}

    \subsubsection{Laves 1980}\label{laves-1980}

    \subsubsection{Steinhardt 1983}\label{steinhardt-1983}

\textbf{\emph{Bond-Orientational Order in Liquids and Glasses}}

Bond-orientational order in molecular dynamics simulations of
supercooled liquids and in models of metallic glasses is studied.
Quadratic and third-order invariants formed from bond spherical
harmonics allow quantitative measures of cluster symmetries in these
systems. A state with short-range translational order, but extended
correlations in the orientations of particle clusters, starts to develop
about 10\% below the equilibrium melting temperature in a supercooled
Lennard-Jones liquid. The order is predominantly icosahedral, although
there is also a cubic component which we attribute to the periodic
boundary conditions. Results are obtained for liquids cooled in an
icosahedral pair potential as well. Only a modest amount of
orientational order appears in a relaxed Finney dense-random-packing
model. In contrast, we find essentially perfect icosahedral bond
correlations in alternative "amorphon" cluster models of glass
structure.

    \subsubsection{Parthé 1984}\label{parthuxe9-1984}

\textbf{\emph{The Standardization of Inorganic Crystal-Structure Data}}

This paper describes a proposal for a standardized presentation of
inorganic crystal-structure data with the aim to recognize identical or
nearly identical structures from the similarity of the numerical values
of the atom coordinates.

\begin{itemize}
\tightlist
\item
  Previous work:

  \begin{itemize}
  \tightlist
  \item
    Schwarzenbach 1963 set of rules for monoclinic and orthorhombic
    structures
  \item
    Niggli 1928 - unique description of a lattice via reduced set of
    basis vectors
  \item
    Buerger 1960 - " "
  \end{itemize}
\item
  Different ways to describe a crystal structure in a standard
  Hermann-Mauguin space group

  \begin{enumerate}
  \def\labelenumi{\arabic{enumi}.}
  \tightlist
  \item
    Shift of origin
  \item
    Rotation
  \item
    Inversion of vector triplet (or reflection of one vector while
    maintaining right-handedness)
  \end{enumerate}
\end{itemize}

Considerations: * Cheshire group / Euclidean normalizer * Symmetry of
arrangement of symmetry elements * Chirality * Centrosymmetry and
Polarity * Calculating non-redundant \emph{xyz} triplets and possible
origins using space group and cheshire group

\begin{itemize}
\tightlist
\item
  Proposal for structure data standardization

  \begin{itemize}
  \tightlist
  \item
    Choice of unit cell and space group (ITC / Niggli)
  \item
    Choice of coordinate triplet for all atoms

    \begin{itemize}
    \tightlist
    \item
      Permitted origins
    \item
      Permitted rotations of coordinate system
    \item
      Enantiomorphic structure representation
    \end{itemize}
  \item
    Ordering and renumbering of atoms
  \end{itemize}
\item
  Benefits and disadvantages

  \begin{itemize}
  \tightlist
  \item
    Benefits

    \begin{itemize}
    \tightlist
    \item
      Unique descriptor of every structuer
    \item
      Allows trivial recognition of isotypy by inspection
    \item
      Allows unequivocal description of unique structures
    \end{itemize}
  \item
    Disadvantages

    \begin{itemize}
    \tightlist
    \item
      Slight deformation masks similarity when standardized
    \end{itemize}
  \end{itemize}
\end{itemize}

    \subsubsection{Gelato 1987}\label{gelato-1987}

\textbf{\emph{STRUCTURE TIDY}} \emph{A Computer Program to Standardize
Crystal Structure Data}

A computer program has been written for the purpose of standardizing
crystal structure data according to rules formulated by Parthé \& Gelato
1984. From input consisting of space-group symbol, unit-cell parameters
and positional coordinates of the atoms, a reordered and renumbered list
of standardized atom coordinates is obtained. The space group is now in
the standard setting and the cell is reduced if applicable. The origin
and the orientation of the coordinate system have been chosen in such a
way as to minimize the standardization parameter \(\Gamma\). A second
standardization parameter CG, based on the position of the centre of
gravity of the atoms in the asymmetric unit, is introduced. The Wyckoff
sequence, obtained from the standardized structure data, can be used to
recognize structures which are isopointal. An example of the application
of STRUCTURE TIDY is given.

    \subsubsection{Lima de Faria 1990}\label{lima-de-faria-1990}

\textbf{\emph{Nomenclature of Inorganic Structure Types}}

Different degrees of similarity between inorganic crystal structures are
defined concisely and examples are presented that illustrate their
practical application. A notation giving the coordination of atoms is
presented together with some basic rules for developing crystal-chemical
formulae and the \emph{Bauverband} description of inorganic structure
types.

\begin{itemize}
\tightlist
\item
  Terms of similarity

  \begin{itemize}
  \tightlist
  \item
    Isopointal

    \begin{itemize}
    \tightlist
    \item
      Same space group type or enantiomorphic spacegroup types
    \item
      Atomic positions are the same after standardization

      \begin{itemize}
      \tightlist
      \item
        i.e. all Wyckoff positions and occupations are identical
      \end{itemize}
    \item
      Structures may need to have origin shifted, be rotated, or be
      permuted

      \begin{itemize}
      \tightlist
      \item
        See affine normalizers in ITC or Parthé and Gelato 1984 and
        Gelato and Parthé 1987
      \end{itemize}
    \end{itemize}
  \item
    Isoconfigurational

    \begin{itemize}
    \tightlist
    \item
      Isopointal
    \item
      For all corresponding Wychoff positions

      \begin{itemize}
      \tightlist
      \item
        Similar crystallographic point configurations (orbits)
      \item
        Similar geometrical interrelationships
      \item
        Note: similarity described later
      \end{itemize}
    \end{itemize}
  \item
    Crystal-Chemically Isotypic (Isostructural)

    \begin{itemize}
    \tightlist
    \item
      Isoconfigurational
    \item
      Corresponding atoms and bonds (interactions) are
      physically/chemically similar
    \end{itemize}
  \item
    Type and Antitype

    \begin{itemize}
    \tightlist
    \item
      Isoconfigurational
    \item
      Some important physical/chemical characteristics of corresponding
      atoms are interchanged (reversed)
    \end{itemize}
  \item
    Homeotypic \textbf{(one or more are relaxed from Isotypism)} (narrow
    Laves 1980)

    \begin{itemize}
    \tightlist
    \item
      Identical or enantiomorphic space group types allowing for

      \begin{itemize}
      \tightlist
      \item
        group-subgroup relationships
      \item
        group-supergroup relationships
      \end{itemize}
    \item
      Limitations imposed on the similarity of geometric properties

      \begin{itemize}
      \tightlist
      \item
        axial ratios
      \item
        interaxial angles
      \item
        values of adjustable positional parameters
      \item
        coordination of corresponding atoms
      \end{itemize}
    \item
      Site occupancy limits

      \begin{itemize}
      \tightlist
      \item
        allow given sites to be occupied by different atomic species
      \end{itemize}
    \end{itemize}
  \item
    Polytypic

    \begin{itemize}
    \tightlist
    \item
      See Guinier \emph{et al.} 1984
    \end{itemize}
  \item
    Interstitial (or '\emph{stuffed}')

    \begin{itemize}
    \tightlist
    \item
      Unoccupied interstitial sites are progressively filled
    \end{itemize}
  \item
    Recombination

    \begin{itemize}
    \tightlist
    \item
      Topologically similar parent structures are periodically divided

      \begin{itemize}
      \tightlist
      \item
        blocks (finite)
      \item
        rods (infinite in one direction)
      \item
        slabs (infinite in two directions)
      \end{itemize}
    \item
      e.g.

      \begin{itemize}
      \tightlist
      \item
        unit cell twinning
      \item
        crystallographic shear planes
      \item
        intergrowth of blocks, rods, or slabs of different structure
        types
      \item
        periodic out-of-phase or antiphase boundaries
      \item
        rotation of rods (blocks)
      \item
        vernier principle (Makovicky and Hyde 1981)
      \end{itemize}
    \end{itemize}
  \end{itemize}
\end{itemize}

    \subsubsection{Parthé 1993}\label{parthuxe9-1993}

\textbf{\emph{Standardization of Crystal Structure Data as an Aid to the
Classification of Crystal Structure Types}}

Owing to the different ways in which crystal structures may be
described, isotypic compounds are often not identified as such. To
remedy this situation, crystal structure data can be standardized by
means of the STRUCTURE TIDY program. In the standardized data of
isotypic structures, occupied sites have the same Wyckoff
representation. This makes it possible to use the Wyckoff sequence (the
letters of occupied Wyckoff sites) to classify crystal structure types.
This classification is much finer than the previously used
classification based on the Pearson code and is of great help if one
wants to know whether a particular atom arrangement is already known.
The standardization has enabled us not only to demonstrate new cases of
isotypism, but also to discover structural relationships between
different structure types with the same space group, for example
substitution, vacancy or filled-in variants.

\begin{itemize}
\tightlist
\item
  Definition of isotypism

  \begin{itemize}
  \tightlist
  \item
    Configuration isotypism

    \begin{itemize}
    \tightlist
    \item
      same stoichiometry (fully ordered structures)
    \item
      same space group
    \item
      same Wyckoff sites
    \item
      same or similary positional (x, y, z)
    \item
      same or similar axial ratios (c/a, a/b, b/c)
    \item
      same or similar cell angles (\(\alpha, \beta, \gamma\))
    \end{itemize}
  \end{itemize}
\end{itemize}

First published structure type compilation \emph{Strukturberichte}
Strukturberichte, suppl, to Z. Kristallogr., Vols. 1-7, Akademische
Verlagsgesellschaft, Leipzig, 1931-1943.

Review of standardization procedure from Parthé 1984

\begin{itemize}
\tightlist
\item
  The same or similary Wyckoff sequence as an indication of possible
  structural relationships

  \begin{itemize}
  \tightlist
  \item
    in standardized description, all isotypic structures have the same
    Wyckoff sequence
  \item
    subdivision of isopointal structures requires detailed study

    \begin{itemize}
    \tightlist
    \item
      roughly approximated by analysis of \(\gamma\) and CG values from
      STRUCTURE TIDY
    \end{itemize}
  \item
    order according to space group and Wyckoff sequence

    \begin{itemize}
    \tightlist
    \item
      substitution derivatives w/ same sg and ws are near
    \end{itemize}
  \item
    identical classification codes + cell param. + atom coord. must be
    used
  \end{itemize}
\end{itemize}

    \subsubsection{Blatov 2006}\label{blatov-2006}

\textbf{\emph{A Method for Hierarchical Comparative Analysis of Crystal
Structures}} A geometrical-topological description of crystal structure
as a three-dimen- sional graph with coloured nodes, weighted and
coloured edges is used to generate a hierarchical sequence of the
structure representations. The solid angles of Voronoi--Dirichlet
polyhedra of atoms are used as the edge weights and the nodes and edges
are coloured according to chemical reasons. Two operations are defined
to derive the representations: contracting an atom to other atoms
keeping the local connectivity, and removing an atom together with all
its bonds. The atoms of the crystal structure are called origin,
removed, contracted or target according to their roles in the
operations. Each structure representation is described as a labelled
quotient graph and determined by (i) colours of the graph nodes and
edges, (ii) some level for edge weights, and (iii) an arrangement of
atoms according to their roles. The computer enumeration and topological
comparative analysis of all representations for crystal structures of
any composition and complexity are implemented into the TOPOS program
package. The advantages of the method are shown by the analysis of
typical inorganic compounds and a molecular packing.

\begin{enumerate}
\def\labelenumi{\arabic{enumi}.}
\item
  Color atoms by species (or other considerations)
\item
  Generate Voronoi solids
\item
  Draw colored coordination lines based on interaction type

  \begin{enumerate}
  \def\labelenumii{\alph{enumii}.}
  \tightlist
  \item
    If the coordination line does not pass through its corresponding
    voronoi surface, ignore it
  \end{enumerate}
\item
\end{enumerate}

    \subsubsection{Allman 2007}\label{allman-2007}

\textbf{\emph{The Introduction of Structure Types into the Inorganic
Crystal Structure Database ICSD}}

Both the approach used and the progress made in the assignment of
structure types to the crystal structures contained in the ICSD database
are reported. Extending earlier work, an hierarchical set of criteria
for the separation of isopointal structures into isoconfigurational
structure types is used. It is shown how these criteria, which include
the space group (number), Wyckoff sequence and Pearson symbol,
\emph{c/a} ratio, \(\beta\) ranges, ANX formulae and, in certain cases,
the necessary elements and forbidden elements, may be used to uniquely
identify the representative structure types of the compounds contained
in the ICSD database.

    \subsubsection{Behler 2007}\label{behler-2007}

\textbf{\emph{Generalized Neural-Network Representation of
High-Dimensional Potential-Energy Surfaces}}

The accurate description of chemical processes often requires the use of
computationally demanding methods like density-functional theory (DFT),
making long simulations of large systems unfeasible. In this Letter we
introduce a new kind of neural-network representation of DFT
potential-energy surfaces, which provides the energy and forces as a
function of all atomic positions in systems of arbitrary size and is
several orders of magnitude faster than DFT. The high accuracy of the
method is demonstrated for bulk silicon and compared with empirical
potentials and DFT. The method is general and can be applied to all
types of periodic and nonperiodic systems.

    \subsubsection{Kondor 2007}\label{kondor-2007}

\textbf{\emph{A Novel Set of Rotationally and Translationally Invariant
Features for Images Based on the Non-Commutative Bispectrum}}

We propose a new set of rotationally and translationally invariant
features for image or pattern recognition and classification. The new
features are cubic polynomials in the pixel intensities and provide a
richer representation of the original image than most existing systems
of invariants. Our construction is based on the generalization of the
concept of bispectrum to the three-dimensional rotation group SO(3), and
a projection of the image onto the sphere.

    \subsubsection{Sanville 2008}\label{sanville-2008}

\textbf{\emph{Silicon Potentials using Density Functional Theory-Fitted
Neural Networks}}

We present a method for fitting neural networks to geometric and
energetic datasets. We then apply this method by fitting a neural
network to a set of data generated using the local density approximation
for systems composed entirely of silicon. In order to generate atomic
potential energy data, we use the Bader analysis scheme to partition the
total system energy among the constituent atoms. We then demonstrate the
transferability of the neural network potential by fitting to various
bulk, surface, and cluster systems.*

    \subsubsection{Handley 2009}\label{handley-2009}

\textbf{\emph{Dynamically Polarizable Water Potential Based on Multipole
Moments Trained by Machine Learning}}

It is widely accepted that correctly accounting for polarization within
simulations involving water is critical if the structural, dynamic, and
thermodynamic properties of such systems are to be accurately
reproduced. We propose a novel potential for the water dimer, trimer,
tetramer, pentamer, and hexamer that includes polarization explicitly,
for use in molecular dynamics simulations. Using thousands of dimer,
trimer, tetramer, pentamer, and hexamer clusters sampled from a
molecular dynamics simulation lacking polarization, we train
(artificial) neural networks (NNs) to predict the atomic multipole
moments of a central water molecule. The input of the neural nets
consists solely of the coordinates of the water molecules surrounding
the central water. The multipole moments are calculated by the atomic
partitioning defined by quantum chemical topology (QCT). This method
gives a dynamic multipolar representation of the water electron density
without explicit polarizabilities. Instead, the required knowledge is
stored in the neural net. Furthermore, there is no need to perform
iterative calculations to self- consistency during the simulation nor is
there a need include damping terms in order to avoid a polarization
catastrophe.

    \subsubsection{Bartók 2010}\label{bartuxf3k-2010}

\textbf{\emph{Gaussian Approximation Potentials: The Accuracy of Quantum
Mechanics, without the Electrons}}

We introduce a class of interatomic potential models that can be
automatically generated from data consisting of the energies and forces
experienced by atoms, as derived from quantum mechanical calculations.
The models do not have a fixed functional form and hence are capable of
modeling complex potential energy landscapes. They are systematically
improvable with more data. We apply the method to bulk crystals, and
test it by calculating properties at high temperatures. Using the
interatomic potential to generate the long molecular dynamics
trajectories required for such calculations saves orders of magnitude in
computational cost.

    \subsubsection{Behler 2011}\label{behler-2011}

\textbf{\emph{Atom-Centered Symmetry Functions for Constructing
High-Dimensional Neural Network Potentials}}

Neural networks offer an unbiased and numerically very accurate approach
to represent highdimensional ab initio potential-energy surfaces. Once
constructed, neural network potentials can provide the energies and
forces many orders of magnitude faster than electronic structure
calculations, and thus enable molecular dynamics simulations of large
systems. However, Cartesian coordinates are not a good choice to
represent the atomic positions, and a transformation to symmetry
functions is required. Using simple benchmark systems, the properties of
several types of symmetry functions suitable for the construction of
high-dimensional neural network potential-energy surfaces are discussed
in detail. The symmetry functions are general and can be applied to all
types of systems such as molecules, crystalline and amorphous solids,
and liquids.

    \subsubsection{Rupp 2012}\label{rupp-2012}

\textbf{\emph{Fast and Accurate Modeling of Molecular Atomization
Energies with Machine Learning}}

We introduce a machine learning model to predict atomization energies of
a diverse set of organic molecules, based on nuclear charges and atomic
positions only. The problem of solving the molecular Schrödinger
equation is mapped onto a nonlinear statistical regression problem of
reduced complexity. Regression models are trained on and compared to
atomization energies computed with hybrid density- functional theory.
Cross validation over more than seven thousand organic molecules yields
a mean absolute error of \textasciitilde{}10 kcal/mol. Applicability is
demonstrated for the prediction of molecular atomization potential
energy curves.

    \subsubsection{Bartók et al. 2013}\label{bartuxf3k-et-al.-2013}

\textbf{\emph{On Representing Chemical Environments}}

We review some recently published methods to represent atomic
neighborhood environments, and analyze their relative merits in terms of
their faithfulness and suitability for fitting potential energy
surfaces. The crucial properties that such representations (sometimes
called descriptors) must have are differentiability with respect to
moving the atoms and invariance to the basic symmetries of physics:
rotation, reflection, translation, and permutation of atoms of the same
species. We demonstrate that certain widely used descriptors that
initially look quite different are specific cases of a general approach,
in which a finite set of basis functions with increasing angular wave
numbers are used to expand the atomic neighborhood density function.
Using the example system of small clusters, we quantitatively show that
this expansion needs to be carried to higher and higher wave numbers as
the number of neighbors increases in order to obtain a faithful
representation, and that variants of the descriptors converge at very
different rates. We also propose an altogether different approach,
called Smooth Overlap of Atomic Positions, that sidesteps these
difficulties by directly defining the similarity between any two
neighborhood environments, and show that it is still closely connected
to the invariant descriptors. We test the performance of the various
representations by fitting models to the potential energy surface of
small silicon clusters and the bulk crystal.

\begin{itemize}
\tightlist
\item
  Goal: accurately and uniquely represent atomic environments

  \begin{itemize}
  \tightlist
  \item
    typical implementation: fingerprint
  \item
    often used for constructing interatomic potentials and fitting
    potential energy surfaces (PES)
  \item
    completeness

    \begin{itemize}
    \tightlist
    \item
      complete: system of invariant descriptors q1, 12, ... qm which
      uniqely determines atomic environment up to symmetries

      \begin{itemize}
      \tightlist
      \item
        one-to-one mapping (bijection) between different atomic
        environments and the invariant tuples comprising the
        representation
      \end{itemize}
    \item
      overcomplete: contains spurious descriptors i.e. a proper subset
      (q1, q2, ..., qm) is complete

      \begin{itemize}
      \tightlist
      \item
        assigns potentially many distinct descriptors to a structure
      \item
        guarantees different environments will never have identical
        descriptors
      \item
        function relating representations and structures is a surjection
      \end{itemize}
    \end{itemize}
  \item
    usually, PESs of small molecules are fitted using first-principles
    calculations

    \begin{itemize}
    \tightlist
    \item
      list of pairwise distances
    \item
      transformed (reciprocal, exponential) interatomic distances
    \item
      permuting atoms gives (seemingly) new config.
    \item
      i.e. crucial symmetries missing
    \end{itemize}
  \item
    Braams and Bowman

    \begin{itemize}
    \tightlist
    \item
      polynomials of pairwise distances
    \item
      invariant to permuation of identical atoms
    \item
      computer code to generate for up to 10 atoms
    \item
      number ot atoms must stay constant
    \end{itemize}
  \item
    solids / large clusters

    \begin{itemize}
    \tightlist
    \item
      number of atoms allowed to vary
    \item
      maintain continuous and differentiable descriptor
    \item
      Behler and Parrinello

      \begin{itemize}
      \tightlist
      \item
        symmetry functions
      \item
        Si, Na, ZnO, H2O, ...
      \end{itemize}
    \item
      Bartok et al

      \begin{itemize}
      \tightlist
      \item
        bispectrum to fit multi-body potential
      \item
        crystalline solids, defetcs in diamond
      \end{itemize}
    \item
      Rupp et al

      \begin{itemize}
      \tightlist
      \item
        ordered eigenvalues of the coulomb matrix
      \item
        fit atomization energies for 7000+ small organic molecules
      \end{itemize}
    \end{itemize}
  \item
    disentangle from first-principles and PES fitting, focus on
    descriptor

    \begin{itemize}
    \tightlist
    \item
      bond-order parameters by Steinhardt et al
    \item
      show they are a subset of invariants called the bispectrum
    \item
      bispectrum formally infinite

      \begin{itemize}
      \tightlist
      \item
        overcomplete
      \item
        truncate for chosen precision
      \end{itemize}
    \item
      relate bispectrum to Behler et al
    \end{itemize}
  \end{itemize}
\item
  PES fitting

  \begin{itemize}
  \tightlist
  \item
    interest in nonparametric PES (i.e. no closed form)
  \item
    calculate a set of training configurations
  \item
    fit \(E_{short} = \sum_n \epsilon(q_1^{(n)},...,q_M^{(n)})\)
  \item
    how to fit \(\epsilon\) ?

    \begin{itemize}
    \tightlist
    \item
      linear fit
    \item
      Behler and Parrniello -\textgreater{} artificial neural networks
    \item
      Bartók et al. -\textgreater{} Gaussian Approximation Potentials
    \item
      all produce
      \(\epsilon(\mathbf{q}) = \sum_{k=1}^N \alpha_k K(\mathbf{q},\mathbf{q}^{(k)})\)

      \begin{itemize}
      \tightlist
      \item
        N number of training configs indexed by k
      \item
        vector \(\mathbf{\alpha}\) fitted
      \item
        K is a fixed nonlinear kernel function

        \begin{itemize}
        \tightlist
        \item
          K is positive definite
          \(K(\mathbf{q}, \mathbf{q'}) = K(\mathbf{q'}, \mathbf{q})\)
        \item
          for nonzero \(\mathbf{\alpha}\),
          \(\sum_k \sum_l \alpha_k \alpha_l K(\mathbf{q}^{(k)}, \mathbf{q}^{(l)}) > 0\)
        \item
          Rasmussen and Williams present some kernels
        \end{itemize}
      \end{itemize}
    \end{itemize}
  \end{itemize}
\item
  Distance metric (non-differentiable :( )

  \begin{itemize}
  \tightlist
  \item
    TODO: summarize
  \end{itemize}
\item
  SOAP (continuous and differentiable!)

  \begin{itemize}
  \tightlist
  \item
    TODO: summarize
  \end{itemize}
\end{itemize}

    \subsubsection{Yang 2014}\label{yang-2014}

\textbf{\emph{Proposed Definition of Crystal Substructure and
Substructural Similarity}}

There is a clear need for a practical and mathematically rigorous
description of local structure in inorganic compounds so that structures
and chemistries can be easily compared across large data sets. Here a
method for decomposing crystal structures into substructures is given,
and a similarity function between those substructures is defined. The
similarity function is based on both geometric and chemical similarity.
This construction allows for large-scale data mining of substructural
properties, and the analysis of substructures and void spaces within
crystal structures. The method is validated via the prediction of Li-ion
intercalation sites for the oxides. Tested on databases of known
Li-ion-containing oxides, the method reproduces all Li-ion sites in an
oxide with a maximum of 4 incorrect guesses 80\% of the time.

    \subsubsection{De et al. 2016}\label{de-et-al.-2016}

\textbf{\emph{Comparing Molecules and Solids Across Structural and
Alchemical Space}}

Evaluating the (dis)similarity of crystalline, disordered and molecular
compounds is a critical step in the development of algorithms to
navigate automatically the configuration space of complex materials. For
instance, a structural similarity metric is crucial for classifying
structures, searching chemical space for better compounds and materials,
and driving the next generation of machine-learning techniques for
predicting the stability and properties of molecules and materials. In
the last few years several strategies have been designed to compare
atomic coordination environments. In particular, the smooth overlap of
atomic positions (SOAPs) has emerged as an elegant framework to obtain
translation, rotation and permutation-invariant descriptors of groups of
atoms, underlying the development of various classes of machine-learned
inter-atomic potentials. Here we discuss how one can combine such local
descriptors using a regularized entropy match (REMatch) approach to
describe the similarity of both whole molecular and bulk periodic
structures, introducing powerful metrics that enable the navigation of
alchemical and structural complexities within a unified framework.
Furthermore, using this kernel and a ridge regression method we can
predict atomization energies for a database of small organic molecules
with a mean absolute error below 1 kcal mol\(^-1\), reaching an
important milestone in the application of machine-learning techniques
for the evaluation of molecular properties.

    \subsubsection{Zhu 2016}\label{zhu-2016}

\textbf{\emph{A Fingerprint Based Metric for Measuring Similarities of
Crystalline Structures}}

Measuring similarities/dissimilarities between atomic structures is
important for the exploration of potential energy landscapes. However,
the cell vectors together with the coordinates of the atoms, which are
generally used to describe periodic systems, are quantities not directly
suitable as fingerprints to distinguish structures. Based on a
characterization of the local environment of all atoms in a cell, we
introduce crystal fingerprints that can be calculated easily and define
configurational distances between crystalline structures that satisfy
the mathematical properties of a metric. This distance between two
configurations is a measure of their similarity/dissimilarity and it
allows in particular to distinguish structures. The new method can be a
useful tool within various energy landscape exploration schemes, such as
minima hopping, random search, swarm intelligence algorithms,
andhigh-throughputscreenings.

    \subsubsection{Mehl 2017}\label{mehl-2017}

\textbf{\emph{The AFLOW Library of Crystallographic Prototypes: Part 1}}

An easily available resource of common crystal structures is essential
for researchers, teachers, and students. For many years this was
provided by the U.S. Naval Research Laboratory's \emph{Crystal Lattice
Structures} web page, which contained nearly 300 crystal structures,
including a majority of those which were given Strukturbericht
designations. This article presents the updated version of the database,
now including 288 standardized structures in 92 space groups. Similar to
what was available on the web page before, we present a complete
description of each structure, including the formulas for the primitive
vectors, all of the basis vectors, and the AFLOW commands to generate
the standardized cells. We also present a brief discussion of crystal
systems, space groups, primitive and conventional lattices, Wyckoff
positions, Pearson symbols and \emph{Strukturbericht} designations.

    \subsubsection{Su 2017}\label{su-2017}

\textbf{\emph{Construction of Crystal Structure Prototype Database:
Methods and Applications}}

Crystal structure prototype data have become a useful source of
information for materials discovery in the fields of crystallography,
chemistry, physics, and materials science. This work reports the
development of a robust and efficient method for assessing the
similarity of structures on the basis of their interatomic distances.
Using this method, we proposed a simple and unambiguous definition of
crystal structure prototype based on hierarchical clustering theory, and
constructed the crystal structure prototype database (CSPD) by filtering
the known crystallographic structures in a database. With similar
method, a program structure prototype analysis package (SPAP) was
developed to remove similar structures in CALYPSO prediction results and
extract predicted low energy structures for a separate theoretical
structure database. A series of statistics describing the distribution
of crystal structure prototypes in the CSPD was compiled to provide an
important insight for structure prediction and high-throughput
calculations. Illustrative examples of the application of the proposed
database are given, including the generation of initial structures for
structure prediction and determination of the prototype structure in
databases. These examples demonstrate the CSPD to be a generally
applicable and useful tool for materials discovery.


    % Add a bibliography block to the postdoc
    
    
    
    \end{document}
